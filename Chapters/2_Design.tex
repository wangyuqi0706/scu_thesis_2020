\chapter{为可理解性设计}
在Raft的设计上,我们有几个目标:它必须为系统构建提供完整且实用的基础设施,
所以必须显著减少开发者的设计工作;它必须在任何情况下保证安全性,在正常情况下
保证可用性;常用操作必须高效。然而我们最重要的目标,也是最大挑战是——可理解性。
他必须有被大量读者轻松地理解的可能性。此外,他必须能让人建立起对算法的直观认识,
这样系统构建者才能对其进行扩展,这在现实世界中十分常见。

在Raft的设计中的很多方面,我们不得不在很多备选设计中选择。在这种情况下
我们优先保持可理解性:各个方案是不是难以解释(比如这个方案的状态有多么复杂,
,它有一些微妙的含义吗?)以及对读者来说,这个方案是不是便于读者完全理解其过程和意义。

我们承认这在分析中是高度主观的,尽管如此,我们使用了两种比较合适的手法。
第一个手法是众所周知的问题拆分:尽可能将一个问题中可以相对独立地解决、解释、理解的
部分拆分成几块。例如,我们将Raft拆分为了leader选举,日志复制,安全性,以及
成员关系变更等部分。

我们的第二个措施是通过减少需要考虑的状态数量,让系统更加条理清晰并尽可能消除
不确定性等方法来简化状态空间。具体来说,日志不允许出现空洞、Raft也限制了
日志变得不一致的途径。尽管我们尝试在大多数情况下消除不确定性,但是在一些情况下
的不确定性确实能改善可理解性。特别地,随机化的方法引入了不确定性,但是有助于
通过用相同的方式处理所有可能的选择,以此减少状态空间。我们使用随机性来简化
Raft的leader选举算法。

\chapter{Raft共识算法}

Raft是一种用于管理第二章所述的复制日志的算法。图二总结了用于参考的浓缩版本
算法,图三算法的关键特性;图中的这些元素将在剩下的部分中分段叙述。

Raft实现共识的方法是:首先,选出一个与众不同的leader,然后把管理日志的全部
责任交给leader。Leader从客户端收到日志条目,将他们复制到其他服务器上,然后
在可以安全地将日志应用到状态机上时通知其他服务器。拥有一个leader可以简化
对复制日志的管理。比如leader可以决定把新的条目放置在哪里,而不需要咨询其他
服务器,而数据流只可以只从leader流向其他服务器。一个leader也可以故障或者
与其他服务器